\section{Introduction}
\label{Introduction}

%Problem statement: What are the deflection, twist and maximum stress experienced by the aileron under a critical loading scenario.

The aileron is a control surface of the aircraft, located towards the tip of the wing. The optimum operation of the aileron at all times is necessary in order to ensure the safety of the passengers and the crew during flight. Thus, correct functioning of the aileron is required even in a critical loading scenario. 
The critical loading scenario that is analysed consists of the aileron being at maximum upward deflection and the aerodynamic loading on the
wing being at limit load. Furthermore, one of the two actuators of the aileron is jammed. In this case, the aileron is subjected to bending, shear and torsion. 

In order to observe the behaviour of the aileron of the Fokker 100 in this critical loading scenario, a numerical structural analysis tool needs to be developed. The tool will have as outputs the deflection of the hinge line, the twist of the aileron, and the maximum stress experienced by the aileron.

This report represents the simulation plan for the actual implementation, verification and validation of the numerical model. In this report, the description of the loading cases is presented first. Then, the theory behind the numerical model as well as the approach regarding its implementation are described. Next, the verification model is discussed, followed by the verification strategy proposed for the team's own numerical model. After this, the validation approach of the constructed model is discussed. The task division as well as a Gantt chart may be found in the Appendices.   
